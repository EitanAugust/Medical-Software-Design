\documentclass[10pt]{report}
\usepackage{epsf}
\usepackage{amsmath}
\usepackage{amssymb}
\usepackage{palatino}
\usepackage[dvips]{graphics}
\usepackage{fancyhdr}
\usepackage{epsfig}
\usepackage{multirow}
\usepackage{multicol}
\usepackage{cancel}
\usepackage{hyperref}
\usepackage{longtable}
\parindent 0in
\parskip 1ex
\oddsidemargin  0in
\evensidemargin 0in
\textheight 8.5in
\textwidth 6.5in
\topmargin -0.25in

\pagestyle{fancy}
\lhead{\bf BME590.06: Medical Software Design}
\rhead{\bf Palmeri \& Kumar (Fall 2017)}
\cfoot{\thepage}

\usepackage{listings}
\usepackage{color}

\definecolor{dkgreen}{rgb}{0,0.6,0}
\definecolor{gray}{rgb}{0.5,0.5,0.5}
\definecolor{mauve}{rgb}{0.58,0,0.82}

\lstset{
  language=Java,
  aboveskip=3mm,
  belowskip=3mm,
  showstringspaces=false,
  columns=flexible,
  basicstyle={\small\ttfamily},
  numbers=none,
  numberstyle=\tiny\color{gray},
  keywordstyle=\color{blue},
  commentstyle=\color{dkgreen},
  stringstyle=\color{mauve},
  breaklines=true,
  breakatwhitespace=true,
  tabsize=3
}

\begin{document}
\section*{Assignment \#03: CloudECG}

{\bf DUE:} Thursday, 2017-11-2 at 15:05.

\subsection*{Summary}

In this assignment, you will create and deploy a well-tested web service that performs heart rate calculations on ECG data along with bradycardia and tachycardia detection. Your web service will be deployed on a public facing Virtual Machine (VM) as it would be in industry, allowing your service to process requests from any internet connected client (e.g. cloud connected ECG device, iOS/Android applications, Web applications, etc).

\subsection*{Instructions}
Create a web service that implements the below RESTful API routes (a.k.a. endpoints). Please be sure to validate inputs and return the correct HTTP error codes with your responses and errors.
\begin{itemize}
	\item {\bf POST /api/heart\_rate/summary} - responsible for returning instantaneous HR and brady/tachy summary.
	This endpoint takes the following as JSON input:	
	\begin{lstlisting}
	{
		"time": [1, 2, 3, 4, ...],
		"voltage": [20.1, 20, 14, ...],
	}
	\end{lstlisting}
	and returns JSON output in the following form:
	\begin{lstlisting}
	{
		"time": [1, 2, 3, ...],
		"instantaneous_heart_rate": [100, 60, 62, ...],  
		"tachycardia_annotations": [true, false, false, ...],
		"bradycardia_annotations": [false, false, false, ...],
	}
	\end{lstlisting}

	\item {\bf POST /api/heart\_rate/average} - return an array of average heart rate and brady/tachy annotations computed over a specified time interval. 
	This endpoint takes the following as JSON input:	
	\begin{lstlisting}
	{
		"averaging_period": 20, // In seconds
		"time": [1, 2, 3, 4, ...],
		"voltage": [20.1, 20, 14, ...],
	}
	\end{lstlisting}
	and returns JSON output in the following form:
	\begin{lstlisting}
	{
		"averaging_period": 20,
		"time_interval": [1, 2, 3, ...],
		"average_heart_rate": [100, 60, 62, ...],  
		"tachycardia_annotations": [true, false, false, ...],
		"bradycardia_annotations": [false, false, false, ...],
	}
	\end{lstlisting}

	\item {\bf GET /api/requests} - return the total number of requests the service has served since its most recent reboot. You may decide the best way to return this data as JSON.
\end{itemize}

\subsection*{Submission}
\begin{itemize}
	\item As with previous assignments, tag your git repository with an `rc` semver tag (e.g. `v1.0rc1`).
	\item Ensure you have an up-to-date README that documents the endpoints your service provides. Be sure that it also includes a link to virtual machine host where your final server is running and servicing requests.
\end{itemize}

\subsection*{Grading Criteria}
\begin{itemize}
	\item Proper version control usage [2]
	\item Adequate unit test coverage, modularity, continuous integration, and docstrings [2]
	\item Proper adherence to general python conventions (PEP8, snake\_case, spacing around operators) [1]
	\item Achieves functional specifications [3]
	\item Is deployed and able to process API requests [1]

\end{itemize}



\end{document}
